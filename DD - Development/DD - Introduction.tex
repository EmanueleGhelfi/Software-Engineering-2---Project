
\section{Introduction}

\subsection{Purpose}

This software design document describes the architectures and the
system design of Power EnJoy. It's mainly intended for developers
but it has a hierarchical structure. It completes the RASD and defines
the component that led to the satisfaction of the goals previous defined.
It starts from an high level description of the architecture and then
it goes into detail. \\
This document has to identify:
\begin{itemize}
\item Architecture of the system
\item Interactions between components
\item Main algorithms of the system
\end{itemize}
The main purpose is to gain a general understanding of how and why
the system is decomposed and how individual components work together. 

\subsection{Scope}

PowerEnJoy is a software that manages a car sharing service for electric
cars. The aim of this software is to make the reservation of cars
simple and quick. So the system should provide users with real time
information about availability of cars, their status and their positions.After
the reservation, users can directly get their car in pre-defined parking
areas. The service will be accessible only to registered users, giving
some personal information and data needed to the payment. The price
of the ride is computed with a fixed amount of money per minute, displayed
by the car, and finally charged. To avoid useless reservation, that
is to say that a user doesn\textquoteright t pick up the car, the
reservation expires after a fixed time, the car returns available,
and the user is charged with a fee. Cars must be locked in the safe
areas, and only users that have made the reservation can unlock them.
The software has to provide also management functionality for administrators
and operators in order to ensure a simple managing of the system. 

\subsection{Definitions, Acronyms, Abbreviation}
\begin{itemize}
\item RASD: Requirements and Specifications Document
\item DD: Design Document (this document)
\item JSON: JavaScript Object Notation
\item REST: Representational State Transfer
\item JDBC: Java Database Connectivity
\item API: Application Programming Interface
\item JEE: Java Enterprise Edition
\end{itemize}

\subsection{Reference Documents}
\begin{itemize}
\item RASD released before this document.
\item Assignments AA 2016-2017.
\item DD from previous years.
\end{itemize}

\subsection{Document Structure}

$\mathbf{Introduction}$: this is a general overview of the document.\\
The $\mathit{Purpose}$ part describe the audience and the main goals
of this document.\\
The $\mathit{Scope}$ part has to provide a description and scope
of the software and explain the goals, objectives and benefits of
the project.\\
$\mathit{Reference}$ $\mathit{Documents}$ are previous documents
of this project and documents used as examples and reference.\\
$\mathbf{\mathbf{Architectural}}$ $\mathbf{Design}$: this section
explains the relationship between the modules to achieve the complete
functionality of the system (requirements defined in the RASD).\\
It contains an high level overview of how responsibilities of the
system were partitioned and then assigned to subsystem (components).
\\
In this part of the document are identified each high level subsystem
and the roles or responsibility assigned to it in order to achieve
a more detailed comprehension of the software to be. It's also described
how these components collaborate with each other in order to achieve
desired functionality. There is a focus on the interface provided
by individual components in $\mathit{Component}$ $\mathit{Interfaces}$
section.\\
$\mathit{Deployment}$ $\mathit{View}$ gives a description of how
the software to be it's intended to be deployed.\\
$\mathit{Runtime}$ $\mathit{View}$ gives a description of the interaction
between components in the most important use case of the system.\\
In the section of $\mathit{Selected}$ $\mathit{Architectural}$ $\mathit{Styles}$
$\mathit{and}$ $\mathit{Patterns}$ are described which styles and
patterns have been followed in the realization of the system. There
is a focus on the rationale of these decisions.\\
$\mathbf{Algorithm}$ $\mathbf{Design}$: this section explains the
most important algorithms of the software to be. Pseudo\textendash code
has been used in order to avoid unnecessary implementation details.
\\
$\mathbf{User}$ $\mathbf{Interface}$ $\mathbf{Design}$: this section
refers to the same section in RASD and provides some extensions.\\
$\mathbf{Requirements}$ $\mathbf{Traceability}$: this section describe
how requirements defined in RASD have been mapped to system components
defined in section 2. \\

